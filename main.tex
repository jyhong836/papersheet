\documentclass[11pt]{article}

\title{Paper Startup Sheet}
\author{Junyuan Hong}

\begin{document}

\maketitle

\abstract{
This \LaTeX file helps you organizing your ideas and start a paper with structural note.
Follow the TODO comments in Latex file to finish the sheet.
}

\section{Introduction}

% Background
% What is the application or related data of your method.

% What is the problem?
% e.g., the data has some difficulties for learning to induct.
% e.g., previous methods has some difficulties on the data.

% (Optional) What is your insight on the problem?
% Analyze the source of the problem.
% Analyze the properties of data

% How will you solve the problem? Why?
% Use one sentence to summarize your ideas.
% List points of contribution:
%	1. The basic of the proposal
%	2. The development of the basic
%	3. Further adaptation of the new method
%	4. (opt) experimental results.

\section{Related Work}

% Organize relate works in neat direction
% 1. Method One
%	Pros/Cons
%	You should make sure the shortcomings can be solved in your proposal.

\section{Your method}

% Write step by step

\section{Experiments}

% Design your experiments
% Data set 1
% Data set properties
% What problem do you want to reveal in this experiment?
% How it emphasize the advantage of your method?

\bibliographystyle{IEEEtran}
\bibliography{refs}

\end{document}
